\documentclass[a4paper,11pt]{article}
\usepackage{czech}
\usepackage[utf8x]{inputenc}
%\usepackage{hyperref}

\usepackage{hyperref}

\parindent=0in

\title{KBang -- zadání}
\author{Michal Čevora}
\date{14. 3. 2008}

\begin{document}

\maketitle

\section*{Obecné informace}

\textbf{KBang} je (bude) síťová implementace karetní hry Bang. Hra Bang je karetní hra pro 4 až 7 hráčů. Každému hráči
je na začátku hry přiřazena jedna z rolí \textit{šerif}, \textit{renegád}, \textit{bandita} a \textit{pomocník}. Tyto role
jsou rozdány anonymně, tedy hráči navzájem neznají své role. Výjimku tvoří \textit{šerif}, jehož role je veřejná.

Každý hráč má dle své role stanoven cíl:
\begin{itemize}
\item \textit{Šerif} musí zabít všechny bandity a renegáda.
\item \textit{Bandité} musí zabít šerifa.
\item \textit{Pomocníci} pomáhají šerifovi.
\item \textit{Renegád} musí zabít všechny ostatní hráče a šerifa jako posledního.
\end{itemize}

Podrobnější pravidla najdete na \url{http://en.wikipedia.org/wiki/Bang!}.

\section*{Detaily}
Aplikace bude napsaná v jazyku \texttt{C++} s využitím knihovny \texttt{Qt 4} (a možná i \texttt{kdelibs}. Aplikace bude rozdělená na
dva subprojekty - klient a server. Tyto dvě komponenty budou spolu komunikovat za pomocí XML proudů, použitých např. v protokolu XMPP.

Serverová část nebude obsahovat umělou inteligenci, ačkoli toto by mohl být prostor pro rozšíření na bakalářskou práci.

Projekt je vyvíjen pod svobodnou licencí GNU/GPL a je hostován na serveru Sourceforge.net pod unixovým názvem kbang.


\end{document}

